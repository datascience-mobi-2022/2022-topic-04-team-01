% Options for packages loaded elsewhere
\PassOptionsToPackage{unicode}{hyperref}
\PassOptionsToPackage{hyphens}{url}
%
\documentclass[
]{article}
\usepackage{amsmath,amssymb}
\usepackage{lmodern}
\usepackage{iftex}
\ifPDFTeX
  \usepackage[T1]{fontenc}
  \usepackage[utf8]{inputenc}
  \usepackage{textcomp} % provide euro and other symbols
\else % if luatex or xetex
  \usepackage{unicode-math}
  \defaultfontfeatures{Scale=MatchLowercase}
  \defaultfontfeatures[\rmfamily]{Ligatures=TeX,Scale=1}
\fi
% Use upquote if available, for straight quotes in verbatim environments
\IfFileExists{upquote.sty}{\usepackage{upquote}}{}
\IfFileExists{microtype.sty}{% use microtype if available
  \usepackage[]{microtype}
  \UseMicrotypeSet[protrusion]{basicmath} % disable protrusion for tt fonts
}{}
\makeatletter
\@ifundefined{KOMAClassName}{% if non-KOMA class
  \IfFileExists{parskip.sty}{%
    \usepackage{parskip}
  }{% else
    \setlength{\parindent}{0pt}
    \setlength{\parskip}{6pt plus 2pt minus 1pt}}
}{% if KOMA class
  \KOMAoptions{parskip=half}}
\makeatother
\usepackage{xcolor}
\IfFileExists{xurl.sty}{\usepackage{xurl}}{} % add URL line breaks if available
\IfFileExists{bookmark.sty}{\usepackage{bookmark}}{\usepackage{hyperref}}
\hypersetup{
  pdftitle={Brain Genes},
  pdfauthor={Joshua Eigenmann},
  hidelinks,
  pdfcreator={LaTeX via pandoc}}
\urlstyle{same} % disable monospaced font for URLs
\usepackage[margin=1in]{geometry}
\usepackage{graphicx}
\makeatletter
\def\maxwidth{\ifdim\Gin@nat@width>\linewidth\linewidth\else\Gin@nat@width\fi}
\def\maxheight{\ifdim\Gin@nat@height>\textheight\textheight\else\Gin@nat@height\fi}
\makeatother
% Scale images if necessary, so that they will not overflow the page
% margins by default, and it is still possible to overwrite the defaults
% using explicit options in \includegraphics[width, height, ...]{}
\setkeys{Gin}{width=\maxwidth,height=\maxheight,keepaspectratio}
% Set default figure placement to htbp
\makeatletter
\def\fps@figure{htbp}
\makeatother
\setlength{\emergencystretch}{3em} % prevent overfull lines
\providecommand{\tightlist}{%
  \setlength{\itemsep}{0pt}\setlength{\parskip}{0pt}}
\setcounter{secnumdepth}{-\maxdimen} % remove section numbering
\ifLuaTeX
  \usepackage{selnolig}  % disable illegal ligatures
\fi

\title{Brain Genes}
\author{Joshua Eigenmann}
\date{2022-06-23}

\begin{document}
\maketitle

\hypertarget{interesting-genes---brain}{%
\section{Interesting genes - brain}\label{interesting-genes---brain}}

Most interesting genes of different brain tissues:

\hypertarget{brain---amygdala}{%
\subsection{Brain - Amygdala}\label{brain---amygdala}}

\hypertarget{enst00000301686}{%
\paragraph{ENST00000301686:}\label{enst00000301686}}

Methyltransferase-like 26\\
\url{https://www.thermofisher.com/order/genome-database/details/gene-expression/Hs04403420_m1}

\hypertarget{enst00000429794}{%
\subsubsection{ENST00000429794:}\label{enst00000429794}}

Cbp/p300 interacting transactivator with Glu/Asp rich carboxy-terminal
domain 1
\url{https://www.thermofisher.com/order/genome-database/details/gene-expression/Hs00918445_g1}

\hypertarget{brain---anterior-cingulate-cortex}{%
\subsection{Brain - Anterior cingulate
cortex}\label{brain---anterior-cingulate-cortex}}

\hypertarget{enst00000337225}{%
\paragraph{ENST00000337225}\label{enst00000337225}}

CXCL14 C-X-C motif chemokine ligand 14 {[} Homo sapiens (human) {]} This
antimicrobial gene belongs to the cytokine gene family which encode
secreted proteins involved in immunoregulatory and inflammatory
processes. The protein encoded by this gene is structurally related to
the CXC (Cys-X-Cys) subfamily of cytokines. Members of this subfamily
are characterized by two cysteines separated by a single amino acid.
This cytokine displays chemotactic activity for monocytes but not for
lymphocytes, dendritic cells, neutrophils or macrophages. It has been
implicated that this cytokine is involved in the homeostasis of
monocyte-derived macrophages rather than in inflammation. {[}provided by
RefSeq, Sep 2014{]} \url{https://www.ncbi.nlm.nih.gov/gene/9547}

\hypertarget{enst00000409385}{%
\subsubsection{ENST00000409385}\label{enst00000409385}}

SPATS2L spermatogenesis associated serine rich 2 like {[} Homo sapiens
(human) {]} Enables RNA binding activity. Located in cytosol; nucleolus;
and nucleoplasm. Part of protein-containing complex. {[}provided by
Alliance of Genome Resources, Apr 2022{]}
\url{https://www.ncbi.nlm.nih.gov/gene/?term=ENST00000409385}

\hypertarget{enst00000602349}{%
\paragraph{ENST00000602349}\label{enst00000602349}}

NXPH1 neurexophilin 1 {[} Homo sapiens (human) {]} This gene is a member
of the neurexophilin family and encodes a secreted protein with a
variable N-terminal domain, a highly conserved, N-glycosylated central
domain, a short linker region, and a cysteine-rich C-terminal domain.
This protein forms a very tight complex with alpha neurexins, a group of
proteins that promote adhesion between dendrites and axons. {[}provided
by RefSeq, Jul 2008{]}
\url{https://www.ncbi.nlm.nih.gov/gene/?term=ENST00000602349}

\hypertarget{brain---cerebellar-hemisphere}{%
\subsection{Brain - Cerebellar
Hemisphere}\label{brain---cerebellar-hemisphere}}

\hypertarget{enst000002766}{%
\paragraph{ENST000002766}\label{enst000002766}}

SYBU syntabulin {[} Homo sapiens (human) {]} Syntabulin/GOLSYN is part
of a kinesin motor-adaptor complex that is critical for the anterograde
axonal transport of active zone components and contributes to
activity-dependent presynaptic assembly during neuronal development (Cai
et al., 2007 {[}PubMed 17611281{]}).{[}supplied by OMIM, Mar 2008{]}
\url{https://www.ncbi.nlm.nih.gov/gene/?term=ENST00000276646}

\hypertarget{enst00000356660}{%
\paragraph{ENST00000356660}\label{enst00000356660}}

BDNF brain derived neurotrophic factor {[} Homo sapiens (human) {]} This
gene encodes a member of the nerve growth factor family of proteins.
Alternative splicing results in multiple transcript variants, at least
one of which encodes a preproprotein that is proteolytically processed
to generate the mature protein. Binding of this protein to its cognate
receptor promotes neuronal survival in the adult brain. Expression of
this gene is reduced in Alzheimer's, Parkinson's, and Huntington's
disease patients. This gene may play a role in the regulation of the
stress response and in the biology of mood disorders. {[}provided by
RefSeq, Nov 2015{]}
\url{https://www.ncbi.nlm.nih.gov/gene/?term=ENST00000356660}

\hypertarget{enst00000518312-enst00000521485}{%
\paragraph{ENST00000518312 \&
ENST00000521485}\label{enst00000518312-enst00000521485}}

SNAP91 synaptosome associated protein 91 {[} Homo sapiens (human) {]}
Predicted to enable several functions, including SNARE binding activity;
clathrin binding activity; and phosphatidylinositol binding activity.
Acts upstream of or within regulation of clathrin-dependent endocytosis.
Predicted to be located in several cellular components, including
postsynaptic density; presynaptic endosome; and presynaptic membrane.
Predicted to be extrinsic component of endosome membrane. Predicted to
be active in several cellular components, including Schaffer collateral
- CA1 synapse; cytoplasmic vesicle; and parallel fiber to Purkinje cell
synapse. Predicted to be extrinsic component of presynaptic endocytic
zone membrane. Biomarker of Alzheimer's disease. {[}provided by Alliance
of Genome Resources, Apr 2022{]}
\url{https://www.ncbi.nlm.nih.gov/gene/?term=ENST00000518312}

\hypertarget{enst00000529690}{%
\paragraph{ENST00000529690}\label{enst00000529690}}

SYBU syntabulin {[} Homo sapiens (human) {]} Syntabulin/GOLSYN is part
of a kinesin motor-adaptor complex that is critical for the anterograde
axonal transport of active zone components and contributes to
activity-dependent presynaptic assembly during neuronal development (Cai
et al., 2007 {[}PubMed 17611281{]}).{[}supplied by OMIM, Mar 2008{]}
\url{https://www.ncbi.nlm.nih.gov/gene/?term=ENST00000529690}

\hypertarget{enst00000577440}{%
\paragraph{ENST00000577440}\label{enst00000577440}}

SEPTIN4 septin 4 {[} Homo sapiens (human) {]} his gene is a member of
the septin family of nucleotide binding proteins, originally described
in yeast as cell division cycle regulatory proteins. Septins are highly
conserved in yeast, Drosophila, and mouse, and appear to regulate
cytoskeletal organization. Disruption of septin function disturbs
cytokinesis and results in large multinucleate or polyploid cells. This
gene is highly expressed in brain and heart. Alternatively spliced
transcript variants encoding different isoforms have been described for
this gene. One of the isoforms (known as ARTS) is distinct; it is
localized to the mitochondria, and has a role in apoptosis and cancer.
{[}provided by RefSeq, Nov 2010{]}

\hypertarget{brain---cerebellum}{%
\subsection{Brain - Cerebellum}\label{brain---cerebellum}}

\hypertarget{enst00000551219}{%
\paragraph{ENST00000551219}\label{enst00000551219}}

PTPRR protein tyrosine phosphatase receptor type R {[} Homo sapiens
(human) {]} The protein encoded by this gene is a member of the protein
tyrosine phosphatase (PTP) family. PTPs are known to be signaling
molecules that regulate a variety of cellular processes including cell
growth, differentiation, mitotic cycle, and oncogenic transformation.
This PTP possesses an extracellular region, a single transmembrane
region, and a single intracellular catalytic domain, and thus represents
a receptor-type PTP. Silencing of this gene has been associated with
colorectal cancer. Multiple transcript variants encoding different
isoforms have been found for this gene. This gene shares a symbol
(PTPRQ) with another gene, protein tyrosine phosphatase, receptor type,
Q (GeneID 374462), which is also located on chromosome 12. {[}provided
by RefSeq, May 2011{]}
\url{https://www.ncbi.nlm.nih.gov/gene/?term=ENST00000551219}

\hypertarget{brain---cortex}{%
\subsection{Brain - Cortex}\label{brain---cortex}}

\hypertarget{enst00000249776}{%
\paragraph{ENST00000249776}\label{enst00000249776}}

KNSTRN kinetochore localized astrin (SPAG5) binding protein {[} Homo
sapiens (human) {]} Enables microtubule plus-end binding activity and
protein homodimerization activity. Involved in several processes,
including cellular response to epidermal growth factor stimulus; mitotic
sister chromatid segregation; and regulation of attachment of spindle
microtubules to kinetochore. Located in several cellular components,
including kinetochore; microtubule cytoskeleton; and ruffle. Implicated
in actinic keratosis and skin squamous cell carcinoma. {[}provided by
Alliance of Genome Resources, Apr 2022{]}
\url{https://www.ncbi.nlm.nih.gov/gene/?term=ENST00000249776}

\hypertarget{enst00000357481}{%
\paragraph{ENST00000357481}\label{enst00000357481}}

ACIN1 apoptotic chromatin condensation inducer 1 {[} Homo sapiens
(human) {]} Apoptosis is defined by several morphologic nuclear changes,
including chromatin condensation and nuclear fragmentation. This gene
encodes a nuclear protein that induces apoptotic chromatin condensation
after activation by caspase-3, without inducing DNA fragmentation. This
protein has also been shown to be a component of a splicing-dependent
multiprotein exon junction complex (EJC) that is deposited at splice
junctions on mRNAs, as a consequence of pre-mRNA splicing. It may thus
be involved in mRNA metabolism associated with splicing. Alternatively
spliced transcript variants encoding different isoforms have been
described for this gene. {[}provided by RefSeq, Oct 2011{]}
\url{https://www.ncbi.nlm.nih.gov/gene/?term=ENST00000357481}

\hypertarget{brain---frontal-cortex}{%
\subsection{Brain - Frontal Cortex}\label{brain---frontal-cortex}}

\hypertarget{enst00000579298}{%
\paragraph{ENST00000579298}\label{enst00000579298}}

NUP85 nucleoporin 85 {[} Homo sapiens (human) {]} This gene encodes a
protein component of the Nup107-160 subunit of the nuclear pore complex.
Nuclear pore complexes are embedded in the nuclear envelope and promote
bidirectional transport of macromolecules between the cytoplasm and
nucleus. The encoded protein can also bind to the C-terminus of
chemokine (C-C motif) receptor 2 (CCR2) and promote chemotaxis of
monocytes, thereby participating in the inflammatory response.
Alternative splicing results in multiple transcript variants.
{[}provided by RefSeq, Dec 2014{]}
\url{https://www.ncbi.nlm.nih.gov/gene/?term=ENST00000579298}

\hypertarget{brain---hippocampus}{%
\subsection{Brain - Hippocampus}\label{brain---hippocampus}}

\hypertarget{enst00000439476}{%
\paragraph{ENST00000439476}\label{enst00000439476}}

BDNF brain derived neurotrophic factor {[} Homo sapiens (human) {]} This
gene encodes a member of the nerve growth factor family of proteins.
Alternative splicing results in multiple transcript variants, at least
one of which encodes a preproprotein that is proteolytically processed
to generate the mature protein. Binding of this protein to its cognate
receptor promotes neuronal survival in the adult brain. Expression of
this gene is reduced in Alzheimer's, Parkinson's, and Huntington's
disease patients. This gene may play a role in the regulation of the
stress response and in the biology of mood disorders. {[}provided by
RefSeq, Nov 2015{]}
\url{https://www.ncbi.nlm.nih.gov/gene/?term=\%23+ENST00000439476}

\hypertarget{enst00000543365}{%
\paragraph{ENST00000543365}\label{enst00000543365}}

ARHGAP45 Rho GTPase activating protein 45 {[} Homo sapiens (human) {]}
Predicted to enable GTPase activator activity. Predicted to be involved
in activation of GTPase activity. Located in membrane. {[}provided by
Alliance of Genome Resources, Apr 2022{]}
\url{https://www.ncbi.nlm.nih.gov/gene/?term=\%23+ENST00000543365}

\hypertarget{brain---nucleus-accumbens}{%
\subsection{Brain - Nucleus accumbens}\label{brain---nucleus-accumbens}}

\hypertarget{enst00000314177}{%
\paragraph{ENST00000314177}\label{enst00000314177}}

Meis homeobox 2
\url{https://www.thermofisher.com/taqman-gene-expression/product/Hs01035451_m1?CID=\&ICID=\&subtype=}

\hypertarget{enst00000353414}{%
\paragraph{ENST00000353414}\label{enst00000353414}}

COL11A1 collagen type XI alpha 1 chain {[} Homo sapiens (human) {]} This
gene encodes one of the two alpha chains of type XI collagen, a minor
fibrillar collagen. Type XI collagen is a heterotrimer but the third
alpha chain is a post-translationally modified alpha 1 type II chain.
Mutations in this gene are associated with type II Stickler syndrome and
with Marshall syndrome. A single-nucleotide polymorphism in this gene is
also associated with susceptibility to lumbar disc herniation. Multiple
transcript variants have been identified for this gene. {[}provided by
RefSeq, Nov 2009{]}
\url{https://www.ncbi.nlm.nih.gov/gene/?term=ENST00000353414}

\hypertarget{enst00000531293}{%
\paragraph{ENST00000531293}\label{enst00000531293}}

SLN sarcolipin {[} Homo sapiens (human) {]} Sarcoplasmic reticulum
Ca(2+)-ATPases are transmembrane proteins that catalyze the
ATP-dependent transport of Ca(2+) from the cytosol into the lumen of the
sarcoplasmic reticulum in muscle cells. This gene encodes a small
proteolipid that regulates several sarcoplasmic reticulum
Ca(2+)-ATPases. The transmembrane protein interacts with Ca(2+)-ATPases
and reduces the accumulation of Ca(2+) in the sarcoplasmic reticulum
without affecting the rate of ATP hydrolysis. {[}provided by RefSeq, Jul
2008{]} \url{https://www.ncbi.nlm.nih.gov/gene/?term=ENST00000531293}

\hypertarget{brain---putamen}{%
\subsection{Brain - Putamen}\label{brain---putamen}}

\hypertarget{enst00000539563}{%
\paragraph{ENST00000539563}\label{enst00000539563}}

LSAMP limbic system associated membrane protein {[} Homo sapiens (human)
{]} This gene encodes a member of the immunoglobulin LAMP, OBCAM and
neurotrimin (IgLON) family of proteins. The encoded preproprotein is
proteolytically processed to generate a neuronal surface glycoprotein.
This protein may act as a selective homophilic adhesion molecule during
axon guidance and neuronal growth in the developing limbic system. The
encoded protein may also function as a tumor suppressor and may play a
role in neuropsychiatric disorders. Alternative splicing results in
multiple transcript variants, at least one of which encodes a
preproprotein that is proteolytically processed. {[}provided by RefSeq,
Jan 2016{]}
\url{https://www.ncbi.nlm.nih.gov/gene/?term=\%23+ENST00000539563}

\hypertarget{brain---putamen-1}{%
\subsection{Brain - Putamen}\label{brain---putamen-1}}

\hypertarget{enst00000475226}{%
\paragraph{ENST00000475226}\label{enst00000475226}}

hemoglobin subunit beta
\url{https://www.ensembl.org/Homo_sapiens/Transcript/Summary?db=core;g=ENSG00000244734;r=11:5225464-5227071;t=ENST00000475226}

\hypertarget{enst00000506554-enst00000519042}{%
\paragraph{ENST00000506554 \&
ENST00000519042}\label{enst00000506554-enst00000519042}}

MEF2C myocyte enhancer factor 2C {[} Homo sapiens (human) {]} This locus
encodes a member of the MADS box transcription enhancer factor 2 (MEF2)
family of proteins, which play a role in myogenesis. The encoded
protein, MEF2 polypeptide C, has both trans-activating and DNA binding
activities. This protein may play a role in maintaining the
differentiated state of muscle cells. Mutations and deletions at this
locus have been associated with severe cognitive disability, stereotypic
movements, epilepsy, and cerebral malformation. Alternatively spliced
transcript variants have been described. {[}provided by RefSeq, Jul
2010{]}
\url{https://www.ncbi.nlm.nih.gov/gene/?term=\%23+ENST00000506554}

\hypertarget{enst00000555247}{%
\paragraph{ENST00000555247}\label{enst00000555247}}

IL11RA interleukin 11 receptor subunit alpha {[} Homo sapiens (human)
{]} Interleukin 11 is a stromal cell-derived cytokine that belongs to a
family of pleiotropic and redundant cytokines that use the gp130
transducing subunit in their high affinity receptors. This gene encodes
the IL-11 receptor, which is a member of the hematopoietic cytokine
receptor family. This particular receptor is very similar to ciliary
neurotrophic factor, since both contain an extracellular region with a
2-domain structure composed of an immunoglobulin-like domain and a
cytokine receptor-like domain. Multiple alternatively spliced transcript
variants have been found for this gene. {[}provided by RefSeq, Jun
2012{]}
\url{https://www.ncbi.nlm.nih.gov/gene/?term=\%23+ENST00000555247}

\hypertarget{brain---substantia-nigra}{%
\subsection{Brain - Substantia nigra}\label{brain---substantia-nigra}}

\hypertarget{enst00000537612}{%
\paragraph{ENST00000537612}\label{enst00000537612}}

EMP1 epithelial membrane protein 1 {[} Homo sapiens (human) {]} Involved
in bleb assembly and cell death. Located in plasma membrane. {[}provided
by Alliance of Genome Resources, Apr 2022{]}
\url{http://genome.ucsc.edu/cgi-bin/hgGene?hgg_gene=ENST00000537612.1\&hgg_chrom=chr12\&hgg_start=13211510\&hgg_end=13216773\&hgg_type=knownGene\&db=hg38}
\url{https://www.ncbi.nlm.nih.gov/gene/?term=ENSG00000134531}

\hypertarget{enst00000590261}{%
\paragraph{ENST00000590261}\label{enst00000590261}}

ILF3 interleukin enhancer binding factor 3 {[} Homo sapiens (human) {]}
This gene encodes a double-stranded RNA (dsRNA) binding protein that
complexes with other proteins, dsRNAs, small noncoding RNAs, and mRNAs
to regulate gene expression and stabilize mRNAs. This protein (NF90,
ILF3) forms a heterodimer with a 45 kDa transcription factor (NF45,
ILF2) required for T-cell expression of interleukin 2. This complex has
been shown to affect the redistribution of nuclear mRNA to the
cytoplasm. Knockdown of NF45 or NF90 protein retards cell growth,
possibly by inhibition of mRNA stabilization. In contrast, an isoform
(NF110) of this gene that is predominantly restricted to the nucleus has
only minor effects on cell growth when its levels are reduced.
Alternative splicing results in multiple transcript variants encoding
distinct isoforms.{[}provided by RefSeq, Dec 2014{]}
\url{https://www.ncbi.nlm.nih.gov/gene/?term=ENST00000590261}

\end{document}
